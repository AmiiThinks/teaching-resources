\documentclass[12pt]{article}
\usepackage{fullpage}

\newtheorem{definition}{Definition}
\newtheorem{question}{Question}
\newtheorem{property}{Property}
\newtheorem{proof}{\em Proof}
\newtheorem{derivation}{\em Sketch}
\newtheorem{notation}{Notation}

\newcommand{\comment}[1]{}
\newcommand{\VS}{\mbox{\it VS}}
\newcommand{\WM}{\mbox{\it WM}}
\newcommand{\PCONJ}{\mbox{\it PCONJ}}
\newcommand{\kDNF}{\mbox{\it kDNF}}
\newcommand{\PDISJ}{\mbox{\it PDISJ}}
\newcommand{\DTrt}{\mbox{\it DT}_{r2}}
\newcommand{\DTs}{\mbox{\it DT}_s}
\newcommand{\PP}{{\rm P}}
\newcommand{\EE}{{\rm E}}
\newcommand{\PX}{\PP_{\!\scriptscriptstyle\! X}}
\newcommand{\PXY}{\PP_{\!\scriptscriptstyle\! X\!Y}}
\newcommand{\PYX}{\PP_{\!\scriptscriptstyle\! Y\!|\!X}}
\newcommand{\PYx}{\PP_{\!\scriptscriptstyle\! Y\!|x}}
\newcommand{\seq}[1]{\langle{#1}\rangle}
\newcommand{\RR}{I\!\!R}
\newcommand{\NN}{I\!\!N}
\newcommand{\argmin}{\arg\!\min}
\newcommand{\argmax}{\arg\!\max}
\newcommand{\eg}{{\em e.g.},\ }
\newcommand{\Eg}{{\em E.g.},\ }
\newcommand{\ie}{{\em i.e.},\ }
\newcommand{\Ie}{{\em I.e.},\ }
\newcommand{\cf}{{\em cf.}\ }
\newcommand{\etc}{{\em etc}}
\newcommand{\aka}{{\em a.k.a.}}
\newcommand{\vardef}{\stackrel{\triangle}{=}}
\def\norm [#1]{{\| #1 \|}}
\newcommand{\sign}{\mbox{\rm sign}}
\newcommand{\err}{\mbox{\rm err}}
\newcommand{\rank}{\mbox{\rm rank}}
\newcommand{\cond}{\mbox{\rm cond}}
\newcommand{\vect}{\mbox{\rm vec}}
\newcommand{\tr}{\mbox{\rm tr}}
\newcommand{\set}[1]{{\{#1\}}}
\newcommand{\tnorm}[2]{\|{#1}\|_{#2}}
\newcommand{\normdot}{{\mbox{$\|\!\cdot\!\|$}}}

%\newcommand{\makevector}[1]{{\tilde{#1}}}
\newcommand{\makevector}[1]{{\bf #1}}
\newcommand{\fvec}{{\makevector{f}}}
\newcommand{\evec}{{\makevector{e}}}
\newcommand{\bvec}{{\makevector{b}}}
\newcommand{\rvec}{{\makevector{r}}}
\newcommand{\dvec}{{\makevector{d}}}
\newcommand{\xvec}{{\makevector{x}}}
\newcommand{\qvec}{{\makevector{q}}}
\newcommand{\yvec}{{\makevector{y}}}
\newcommand{\mvec}{{\makevector{m}}}
\newcommand{\vvec}{{\makevector{v}}}
\newcommand{\zvec}{{\makevector{z}}}
\newcommand{\avec}{{\makevector{a}}}
\newcommand{\wvec}{{\makevector{w}}}
\newcommand{\cvec}{{\makevector{c}}}
\newcommand{\Xvec}{{\makevector{X}}}
\newcommand{\Fvec}{{\makevector{F}}}
\newcommand{\Avec}{{\makevector{A}}}
\newcommand{\Bvec}{{\makevector{B}}}
\newcommand{\Hvec}{{\makevector{H}}}
\newcommand{\Lvec}{{\makevector{L}}}
\newcommand{\Mvec}{{\makevector{M}}}
\newcommand{\Nvec}{{\makevector{N}}}
\newcommand{\Vvec}{{\makevector{V}}}
\newcommand{\Uvec}{{\makevector{U}}}
\newcommand{\Ivec}{{\makevector{I}}}
\newcommand{\Ovec}{{\makevector{O}}}
\newcommand{\smallxvec}{{\scriptsize\mathbf x}}
\newcommand{\alphavec}{\mbox{\boldmath $\alpha$}}
\newcommand{\betavec}{\mbox{\boldmath $\beta$}}
\newcommand{\muvec}{\mbox{\boldmath $\mu$}}
\newcommand{\phivec}{{\mbox{\boldmath $\phi$}}}
\newcommand{\lambdavec}{\mbox{\boldmath $\lambda$}}
\newcommand{\Lambdavec}{\mbox{\boldmath $\Lambda$}}
\newcommand{\Sigmavec}{\mbox{\boldmath $\Sigma$}}
\newcommand{\yy}{{\tt y}}
\newcommand{\uu}{{\tt u}}
\newcommand{\zerovec}{{\makevector{0}}}
\newcommand{\smallzerovec}{{\scriptsize\bf 0}}
\newcommand{\smallonevec}{{\scriptsize\bf 1}}
\newcommand{\onevec}{{\makevector{1}}}
\newcommand{\smallbetavec}{\mbox{\scriptsize\boldmath $\beta$}}
\newcommand{\smallmuvec}{\mbox{\scriptsize\boldmath $\mu$}}


\begin{document}

\noindent
{\Large\bf AUCSC 460 -- Artificial Intelligence}

\vspace*{1\baselineskip}

\noindent
{\large\bf Assignment 2: Search and Planning}

\vspace*{1\baselineskip}

\noindent
Winter 2016\\
Department of Science\\
University of Alberta, Augustana Faculty\\
{\bf Instructor}: Anna Koop, HB 1-31, akoop@ualberta.ca

\vspace*{1.75\baselineskip}
\hrule

\begin{tabbing}
{\bf Instructor}:\ \ \=\kill
\>{\bf Due}:\'{\em Friday, Feb 12 at 23:55:00 local time}
\\
\>{\bf Worth}:\' 25\% of final grade \\
\>(5 questions worth 5\% each, plus a 2\% bonus question.)
\end{tabbing}

\hrule

\vspace*{1.25\baselineskip}

\noindent
{\bf Note}:
Please submit your assignment through eclass. Some questions require you to write small programs,
however for the written questions please format your answers
in a single {\em pdf\/} file.
Submit all required documents in a zip or tar archive.


\section{Constraint satisfaction \rm(10\%)}

%%%%%%%%%%%%
\vspace*{1\baselineskip}
{\bf Definition}:
Propositions in {\em conjunctive normal form\/} are of the form:
\begin{eqnarray*}
\mbox{\em CNF} & \leftarrow & 
\mbox{\em clause}_1 
\wedge
\mbox{\em clause}_2 
\wedge
\cdots
\wedge
\mbox{\em clause}_m 
\\
\mbox{\em clause} & \leftarrow & 
\mbox{\em literal}_1 
\vee
\mbox{\em literal}_2 
\vee
\cdots
\vee
\mbox{\em literal}_k 
\\
\mbox{\em literal} & \leftarrow & {\tt p}_i
\\
\mbox{\em literal} & \leftarrow & \neg {\tt p}_i
%\\
%&&
\quad
\mbox{ where } {\tt p}_i \mbox{ is a primitive proposition}
\end{eqnarray*}

\smallskip

\noindent
Propositions in {\em strict conjunctive normal form\/} 
do not have any repeated primitive propositions in clauses,
nor any opposing pair if literals in any clause
(that is, both {\tt p} and $\neg$ {\tt p} never occur in the same clause).

\noindent
In this assignment you will represent:

\begin{description}

\item[Primitive propositions]
by positive integers, 1, 2, 3, etc.

For example, a primitive proposition ${\tt p}_1$ will be denoted by 1, 
${\tt p}_2$ by 2, etc.

\item[Literals]
by nonzero integers, 1, -1, 2, -2, etc.

For example, a negated literal $\neg {\tt p}_2$ will be denoted by -2, etc.

\item[Clauses]
by an array of nonzero integers (literals).

For example, a clause $({\tt p}_1\vee\neg{\tt p}_2)$ will be denoted by [1, -2], etc.

%Note: that an empty clause, i.e. a clause array of length zero, is assumed to
%denote {\em false}.

\item[CNF formulas]
by an array of clauses.

For example, a CNF formula $({\tt p}_1\vee\neg{\tt p}_2)\wedge({\tt p}_2\vee{\tt p}_3)$
will be denoted by [ [1, -2], [2, 3] ].

%Note: that an empty CNF formula, i.e. a formula array of length zero, 
%is assumed to denote {\em true}.

Note: that a clause that contains an opposing pair of literals,
$i$ and -$i$, immediately evaluates to {\em true\/} and can be removed from
the CNF formula.

\item[Truth values] by 1 and -1, where 1 denotes the value {\em true\/} 
and -1 denotes {\em false}. 

\item[Truth value assignments] by an array of 1's, -1's, and 0's.
The zero denotes that no truth value is assigned.

For example, a truth value assignment to a set of primitive propositions
would be an array [-1, 1, ... , -1].  The assignment to primitive proposition
$p_i$ is located in array position $i$.



\end{description}

\bigskip

\noindent
Consider a complete truth value assignment to all primitive propositions.
\begin{itemize}
\item
The assignment {\em satisfies\/} a positive literal $i$,
if $i$ is assigned value 1 in the assignment.
\item
The assignment {\em satisfies\/} a negative literal -$i$,
if $i$ is assigned value -1 in the assignment.
\item
The assignment {\em satisfies\/} a clause if it satisfies
{\em at least one\/} literal in the clause.

(Thus, a length zero clause is unsatisfiable.)
\item
The assignment {\em satisfies\/} a CNF formula if is satisfies
{\em every\/} clause in the formula.

(Thus, a length zero CNF formula is unfalsifiable.)
\end{itemize}

\newpage

\subsection{Generating constraint satisfaction problems \rm(2\%)}
Write a function {\tt randsat(n, k, m)}.
%
Given three positive integers $n$, $k$ and $m$, the function
generates a random CNF formula
\mbox{{\tt C} $=$ [$c_1$, $c_2$, $...$, $c_m$]}
in strict conjunctive normal form,
where each clause $c_j$ contains exactly $k$ random literals
from \{1, -1, 2, -2, ..., $n$, -$n$\}.
In particular, for each clause $c_i$ choose $k$ random primitives
from \{1, 2, ..., $n$\},
and for each primitive, $i$, randomly decide whether to negate it or not.
Use a different seed for the random number generator in each run.

\medskip

\noindent
{\bf Note} that
one can think of the primitive propositions $i$ as {\em variables\/}
that need truth values assigned and of the clauses $c_j$
as {\em constraints\/} that need to be satisfied simultaneously.
You will use this function to generate random constraint satisfaction problems.

\medskip

\noindent
{\bf Interestingly}
for $k=3$ it is known that if $m<3n$ then there are usually not enough
constraints to make the problem unsatisfiable
(that is, most of the random {\tt C}'s will be satisfiable);
whereas if $m>6n$ then there are usually too many constraints
to make the problem satisfiable
(that is, most of the random {\tt C}'s will be unsatisfiable).
A critical ``phase transition'' occurs near $m=4.26n$.
At this point half of the random problems are satisfiable.
Curiously, random problems generated at the phase transition
appear to be the hardest to solve (on average)
for any solution technique.


%%%%%%%%%%%%
\subsection{Simple complete search \rm(3\%)}
Write a function {\tt satcom(C)}.

The function takes a CNF formula {\tt C} 
%(in strict conjunctive normal form) 
and returns an array of truth value assignments 
that satisfies {\tt C}.  The array should have size $(n+1)$ where $p_n$
is the largest-numbered primitive proposition occurring in {\tt C}.
(Notice that most languages index arrays beginning with 0 and we want to index
variables by their name.)
Any array positions corresponding to primitives \emph{not occurring} in {\tt C}
should be initialized to 0.
If there are no assignments that satisfy {\tt C} then {\tt satcom} should return {\tt None} (or equivalent).

\medskip

So, for example:

{\tt satcom( [[-1, 2], [-2, 4]] ) }

should return

\medskip

\hspace*{0.37cm}
{\tt [0, 1, 1, 0, 1]} \hspace{.93cm} or
{\tt [0, -1, 1, 0, 1]}\par or
{\tt [0, -1, -1, 0, 1]} \hspace*{.5cm} or
{\tt [0, -1, -1, 0, -1]}\par or
{\tt [0, 0, 1, 0, 1]} ($p_1$ remains unassigned)\par or
{\tt [0, -1, 0, 0, 1]} ($p_2$ remains unassigned)\par or
{\tt [0, -1, -1, 0, 0]} ($p_4$ remains unassigned).

\medskip

That is, if there are many satisfying assignments, your function
should just return one.

\bigskip

Similarly:

{\tt satcom( [[1], [-1]] )}
should return {\tt NULL}.

\bigskip

\noindent
You must implement a simple backtrack search procedure as outlined below.

\newpage

\subsection*{Pseudo code for a simple backtrack search}

\vspace*{2ex}

%Davis, Logemann, Loveland (1962) A machine program for theorem-proving.
%Communications of the ACM, 7, pp. 201-215.

\begin{tabbing}
{\bf fu}\={\bf nction} {\tt satcom} ({\tt C})
\\
\\
\> {\tt assign} $=$ [0, 0, ..., 0]
\\
\\
\> {\bf if} {\tt C} has length zero, {\bf return} {\tt assign}
\\
\\
\> {\bf else if} 
{\tt C} contains a zero length clause, {\bf return} {\tt NULL}
\\
\\
\> {\bf else if}
\={\tt C} contains a proposition {\tt p} that occurs only in ``{\tt and}'' parts or %\\
\kill
%\>\> as the only proposition in some $impl$ (and in the ``{\tt and}'' part)
%\\
%\>\> $v :=$ {\tt false}
%%\\
%\>\> $r :=$ {\tt satcom}(assign {\tt p} false in {\tt C})
%\\
%\\
%\> {\bf else if}
%\>{\tt C} contains a proposition {\tt p} that occurs only in ``{\tt or}'' 
%parts or
%\\
%\>\> as the only proposition in some $impl$ (and in the ``{\tt or}'' part)
%\\
%\>\> $v :=$ {\tt true}
%\\
%\>\> $r :=$ {\tt satcom}(assign {\tt p} true in {\tt C})
%\\
\> {\bf else}
\\
\>\>
pick any primitive proposition $i$ that occurs in {\tt C}
\\
\>\> {\tt assign}$[i] :=  -1$
\\
\>\> $r :=$ {\tt satcom}(assign $i$ false in {\tt C})
\\
\>\> {\bf if} \=$r=$ {\tt NULL}
\\
\>\>\> {\tt assign}$[i] := 1$
\\
\>\>\> $r :=$ {\tt satcom}(assign $i$ true in {\tt C})
\\
\>{\bf endif}
\\
\\
\> {\bf if} $r=$ {\tt NULL}
\\
\>\> {\bf return} {\tt NULL}
\\
\> {\bf else}
\\
\>\> {\bf return} {\tt assign}
\end{tabbing}

\vspace*{1ex}

\begin{tabbing}
{\bf function}\= ``assign $i$ false in {\tt C}''
\\
\> Temporarily remove all clauses from {\tt C} that contain $-i$
\\
\> Temporarily remove $i$ from remaining clauses
\end{tabbing}

\vspace*{1ex}

\begin{tabbing}
{\bf function}\= ``assign $i$ true in {\tt C}''
\\
\> Temporarily remove all clauses from {\tt C} that contain $i$
\\
\> Temporarily remove $-i$ from remaining clauses
\end{tabbing}

\bigskip

\noindent
{\em Note:}\/ all clauses and literals removed from {\tt C}
have to be restored to their previous state after each
backtrack.

\vspace*{1.75\baselineskip}

%\hrule

\newpage

%%%%%%%%%%%%
\subsection{Faster complete search \rm(3\%)}
Write a function {\tt satcomfast(C)}.
%
As before,
the function takes a CNF formula {\tt C} 
%(in strict conjuctive normal form) 
and returns a truth value assignment
that satisfies {\tt C}.
If there are no such assignments then {\tt satcomfast} 
should return {\tt NULL}.

You must implement the improved backtrack search procedure outlined below.

\subsection*{Pseudo code for an improved backtrack search}

\vspace*{1ex}

\begin{tabbing}
{\bf function} {\tt satcomfast} ({\tt C})
\end{tabbing}

\noindent
Follow the same outline as the simple backtrack search,
except add the following two conditions
between the first ``{\bf else if}'' and ``{\bf else}'':


\begin{tabbing}
{\bf fu}\={\bf nction} {\tt satcomfast} ({\tt C})
\kill
\> {\bf else if}
\={\tt C} contains a proposition $i$ that occurs only negatively \\
\>\> or occurs negatively in some clause of length one 
\\
\>\> {\tt assign}$[i] := -1$
\\
\>\> $r :=$ {\tt satcomfast}(assign $i$ false in {\tt C})
\\
\\
\> {\bf else if}
\>{\tt C} contains a proposition $i$ that occurs only positively \\
\>\> or occurs positively in some clause of length one 
\\
\>\> {\tt assign}$[i] := 1$
\\
\>\> $r :=$ {\tt satcomfast}(assign $i$ true in {\tt C})
\end{tabbing}
 
 \subsection{Discussion \rm(2\%)}
 Describe the performance of your two search implementations in terms of {\tt n, k, m}. Be sure to consider both average case and worst-case results (you may wish to construct a pathological case to force worst-case results).
 
\newpage

\section{State Search \rm(15\%)}


\subsubsection*{The eight-puzzle}

\noindent
The eight puzzle consists of a $3\times 3$ board with numbered 
squares $1,...,8$ and a special {\em blank\/} square.
The blank can be swapped with any square that is immediately above, below, 
to the left, or to the right of it.
The goal state is given by the configuration
%
\begin{center}
$\begin{array}[t]{|c|c|c|}\hline
1 & 2 & 3 \\\hline
8 &   & 4 \\\hline
7 & 6 & 5 \\\hline
\end{array}$
\end{center}

\noindent
An instance of the eight-puzzle problem is specified by giving an
initial state. 
The objective is to find a shortest sequence of
moves that takes the initial state to the goal state.
For example

\[
\begin{array}{|c|c|c|}\hline
1 & 3 & 4 \\\hline
8 &   & 5 \\\hline
7 & 2 & 6 \\\hline
\end{array}
\begin{array}{c}d\\\rightarrow\end{array}
\begin{array}{|c|c|c|}\hline
1 & 3 & 4 \\\hline
8 & 2 & 5 \\\hline
7 &   & 6 \\\hline
\end{array}
\begin{array}{c}r\\\rightarrow\end{array}
\begin{array}{|c|c|c|}\hline
1 & 3 & 4 \\\hline
8 & 2 & 5 \\\hline
7 & 6 &   \\\hline
\end{array}
\begin{array}{c}u\\\rightarrow\end{array}
\begin{array}{|c|c|c|}\hline
1 & 3 & 4 \\\hline
8 & 2 &   \\\hline
7 & 6 & 5 \\\hline
\end{array}
\begin{array}{c}u\\\rightarrow\end{array}
\begin{array}{|c|c|c|}\hline
1 & 3 &   \\\hline
8 & 2 & 4 \\\hline
7 & 6 & 5 \\\hline
\end{array}
\begin{array}{c}l\\\rightarrow\end{array}
\begin{array}{|c|c|c|}\hline
1 &   & 3 \\\hline
8 & 2 & 4 \\\hline
7 & 6 & 5 \\\hline
\end{array}
\begin{array}{c}d\\\rightarrow\end{array}
\begin{array}{|c|c|c|}\hline
1 & 2 & 3 \\\hline
8 &   & 4 \\\hline
7 & 6 & 5 \\\hline
\end{array}
\]

\noindent
In this assignment you will implement 
a random initial state generator, {\tt randep},
and three procedures for solving eight-puzzle problems:
an iterative deepening search, {\tt ids};
an iterative deepening $A^*$ search, {\tt idas};
and a faster version of $\mbox{\it IDA}^*$ search, {\tt idasfast}.

\subsubsection*{Representation}
To implement these search procedures you will need to represent
states and actions in simple form.
For this assignment you will represent states by an array of integers
where the $i$th location of the array contains the number of the
square in the $i$th location of the board. 
The board locations will be indexed
%
$\begin{array}[t]{|c|c|c|}\hline
{\ell}_0 & {\ell}_3 & {\ell}_6 \\\hline
{\ell}_1 & {\ell}_4 & {\ell}_7 \\\hline
{\ell}_2 & {\ell}_5 & {\ell}_8 \\\hline
\end{array}$
Thus, the goal state,
$\begin{array}[t]{|c|c|c|}\hline
1 & 2 & 3 \\\hline
8 &   & 4 \\\hline
7 & 6 & 5 \\\hline
\end{array}$,
will be represented by an array {\tt \{1, 8, 7, 2, 0, 6, 3, 4, 5 \}},
where
the blank square is represented by a {\tt 0}.

\begin{itemize}
\item
The action {\em move-up\/} (``{\tt u}'') is implemented by swapping {\tt 0} 
with
the number one position to the left, as long as
{\tt 0} is not in the $\ell_0$, $\ell_3$ or $\ell_6$ position.
\item
The action {\em move-down\/} (``{\tt d}'') is implemented by swapping {\tt 0} 
with
the number one position to the right of it in the array, as long as
{\tt 0} is not in the $\ell_2$, $\ell_5$ or $\ell_8$ position.
\item
The action {\em move-right\/} (``{\tt r}'') is implemented by swapping {\tt 0} 
with
the number three positions to the right (if possible).
\item
The action {\em move-left\/} (``{\tt l}'') is implemented by swapping {\tt 0} 
with
the number three positions to the left of it in the array (if possible).
\end{itemize}

\comment{
%
$\begin{array}[t]{|c|c|c|}\hline
{\ell}_0 & {\ell}_1 & {\ell}_2 \\\hline
{\ell}_3 & {\ell}_4 & {\ell}_5 \\\hline
{\ell}_6 & {\ell}_7 & {\ell}_8 \\\hline
\end{array}$
Thus, the goal state,
$\begin{array}[t]{|c|c|c|}\hline
1 & 2 & 3 \\\hline
8 &   & 4 \\\hline
7 & 6 & 5 \\\hline
\end{array}$,
will be represented by an array {\tt \{1, 2, 3, 8, 0, 4, 7, 6, 5 \}},
where
the blank square is represented by a {\tt 0}.

\begin{itemize}
\item
The action {\em move-up\/} (``{\tt u}'') is implemented by swapping {\tt 0} with
the number three positions to the left of it in the array (if possible).
\item
The action {\em move-down\/} (``{\tt d}'') is implemented by swapping {\tt 0} with
the number three positions to the right (if possible).
\item
The action {\em move-right\/} (``{\tt r}'') is implemented by swapping {\tt 0} with
the number one position to the right of it in the array, as long as
{\tt 0} is not in the $\ell_2$, $\ell_5$ or $\ell_8$ position.
\item
The action {\em move-left\/} (``{\tt l}'') is implemented by swapping {\tt 0} with
the number one position to the left, as long as
{\tt 0} is not in the $\ell_0$, $\ell_3$ or $\ell_6$ position.
\end{itemize}
}

%%%%%%%%%%%%
\subsection{Generating eight-puzzle problems \rm(1\%)}

Write a function {\tt randep(d)}
which, for an argument \verb+d+$ \ge 0$, generates a random
initial state by starting with the goal state and taking $d$ random
(legal) moves from the goal.
The function should return an array representing the state it reaches.

%%%%%%%%%%%%
\subsection{Iterative deepening search \rm(5\%)}

Write a function {\tt ids(s, g)}.
%
The function takes two arguments---\verb+s+, an initial state, and
\verb+g+, a goal state---and returns a shortest list of moves
{\tt [$m_1$, $m_2$, \ldots $m_k$ ]}
that takes the initial state \verb+s+ to the goal \verb+g+.
Each move $m_i$ must be one of characters {\tt l}, {\tt r}, {\tt u},
or {\tt d}.
Specifically you must implement an {\em iterative deepening search\/}
to do this.
(You can assume a solution exists.)

\medskip

\noindent
Thus,
%
\begin{verbatim}
	s = [1,8,7,3,0,2,4,5,6]
	g = [1,8,7,2,0,6,3,4,5]
	r = ids.ids(s, g);
\end{verbatim}
\comment{
\begin{verbatim}
	s = [1,3,4,8,0,5,7,2,6]
	g = [1,2,3,8,0,4,7,6,5]
	r = Ids.ids(s, g)
\end{verbatim}
}
%
should return
%
\begin{verbatim}
	r = [ 'd', 'r', 'u', 'u', 'l', 'd' ]
\end{verbatim}

 
%%%%%%%%%%%%
\subsection{Iterative deepening $A^*$ search \rm(6\%)}

Write a  function {\tt idas(s, g)}
%
The precondition and postcondition for this procedure
are the same as for \verb+ids+.
However, for this part you must implement the problem solver by using an
{\em iterative deepening $A^*$ search}.
Recall that
{\em ID}$A^*$ requires a heuristic function $\hat{h}(s)$ for estimating
the number of steps remaining from a state $s$ to a goal $g$.
Use the following heuristic
\[
\hat{h}(s) = \sum_{\mbox{pieces}} \mbox{ distance from piece's location in $s$
to its target location in $g$}
\]
where the distance is calculated by determining the shortest (u,d,l,r) path 
from a piece's location in $s$ to its goal location in $g$.

%\newpage


%%%%%%%%%%%%
\subsection{Faster {\em ID}$A^*$ search \rm(3\%)}

Write a function {\tt idasfast(s, g)}.
%
Implement an efficient method for pruning additional states 
that results in a significantly faster search than {\tt idas}.
Come up with an idea for doing this on your own, 
and clearly document the method you use.

\newpage


\end{document}
